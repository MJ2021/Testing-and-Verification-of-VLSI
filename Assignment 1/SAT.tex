\documentclass[12pt]{article}
\usepackage{graphicx}
\usepackage{float}
\usepackage{graphicx}
\usepackage{subfig} 
\usepackage{subcaption}
\captionsetup{compatibility=false}
\usepackage{placeins}
\usepackage{psfrag}
\usepackage{listings}
\usepackage{amsmath}
\usepackage{wrapfig}
\usepackage{helvet}
\usepackage{float}
\usepackage{float}
\usepackage{siunitx} 


\title{Testing and Verification of VLSI\\EE709}
\author{Mohit\\20D070052 }
\date{April 2023}

\begin{document}

\maketitle
\begin{figure}[H]
\begin{center}
\includegraphics[scale = 0.2]{LOGO.jpeg}
\end{center}
\end{figure}
\section{Student Details}
\begin{tabular}{ l l  }
 Name: & Mohit \\ 
 Roll No: & 20D070052  \\  
\end{tabular}

\section{Question 1}


\subsection{Calculating the product of sums formula (CNF) which describes this network.}
%Insert image

CNF (product of sums formula)form is given below :\\
\[K = (\overline{\overline{AB}+\overline{BC}})\oplus(\overline{\overline{BC}+\overline{CD}})\]
As it is XOR, we can also write it as-
\[(\overline{AB}+\overline{BC})\oplus(\overline{BC}+\overline{CD})\]
\[((\overline{A}+\overline{B})+(\overline{B}+\overline{C}))\oplus((\overline{B}+\overline{C})+(\overline{C}+\overline{D}))\]
\[(\overline{A}+\overline{B}+\overline{C}) \oplus (\overline{B}+\overline{C}+\overline{D})\]
\[\overline{A}BCD + ABC\overline{D}\]
\[BC(\overline{A}D+A\overline{D})\]
\[B.C.(A+D).(\overline{A} + \overline{D})\]
CNF is - 
\[K = B.C.(A + D).(\overline{A} + \overline{D})\]

\subsection{Using the minisat solver, find an input assignment such that the
output K is 0}
To solve the above, we will find the negation CNF form of the above to get 
\[\overline{K} = \overline{\overline{A}BCD + ABC\overline{D}}\]
\[\overline{K} = (A+\overline{B}+\overline{C}+\overline{D}).(\overline{A}+\overline{B}+\overline{C}+D)\]

This CNF form in the minisat solver with the following code:
\begin{verbatim}
c
c   nvars nclauses
p cnf 4 2
c
c   -2 means (~x2), 2 means x2
c   A = x1, B = x2, C = x3, D = x4
c
c   x1 + x2 + x3 + ~x4
1 2 3 -4 0
c
c   ~x1 + x2 + x3 + x4
-1 2 3 4 0
\end{verbatim}

Result - 

SAT\\
-1 -2 -3 -4 0\\

The equation K = 0 can be satisfied using -
\[\overline{A},\overline{B},\overline{C},\overline{D}\]
which is, as given by SAT solver
\[A = 0, B = 0, C = 0, D = 0\]

\subsection{Using the minisat solver, find an input assignment such that the
output K is 1.}
We insert the above CNF form of K in the minisat solver with the following code:
\[K = B.C.(A + D).(\overline{A} + \overline{D})\]
\begin{verbatim}
c
c   nvars nclauses
p cnf 4 4
c.
c   -2 means (~x2), 2 means x2
c   A = x1, B = x2, C = x3, D = x4
c
c   x2
2 0
c
c   x3
3 0
c
c   x1 + x4
1 4 0
c
c   (~x1) + (~x4)
-1 -4 0
\end{verbatim}

Result -
SAT\\
-1 2 3 4 0\\

The equation K = 0 can be satisfied using SAT solver -
\[\overline{A}, B, C, D\]
which is,
\[A = 0, B = 1, C = 1, D = 1\]

\section{Question 2}


\subsection{Method}
Suppose the machine starts in the state \(P=Q=R=S=1\). Using minisat, finding a sequence of input values at \(X\) which will take the machine to the state \(P=Q=R=S=0\).\\
\\
Let \(P(k), Q(k), R(k), S(k)\) be the state variables at time \(k\). Also, let \(X(k)\) be the input at time \(k\). Note that \(Y(k)=P(k) \oplus S(k)\).\\
\\
To solve the problem, we observe:\\
\\
1. At \(k=0, P(k)=Q(k)=R(k)=S(k)=1\). That is, at \(k=0\), the following function is true.
\[
C_{0}=P(0) . Q(0) . R(0) . S(0)
\]
2. At some time \(N\), we want \(P(N)=Q(N)=R(N)=S(N)=0\). That is, at time \(N\), the following function is true:
\[
D_{N}=\overline{P(N)} \cdot \overline{Q(N)} \cdot \overline{R(N)} \cdot \overline{S(N)}
\]
Further, we have -
\[
\begin{aligned}
P(k+1) & =X(k) \oplus P(k) \oplus S(k) \\
Q(k+1) & =P(k) \\
R(k+1) & =Q(k) \\
S(k+1) & =R(k)
\end{aligned}
\]
To find the solution, we proceed as follows. At any \(k>0\), the relation between the state variables at time \(k\) and those at time \(k-1\) is described by the function
\[
\begin{aligned}
\phi_{k}= & (P(k)=X(k-1) \oplus P(k-1) \oplus S(k-1)) \\
\cdot & (Q(k)=P(k-1)) \\
\cdot & (R(k)=Q(k-1)) \\
\cdot & (S(k)=R(k-1))
\end{aligned}
\]
Thus, to see if it possible to find an input sequence which takes us to state \(P(N)=Q(N)=R(N)=S(N)=0\) at some time \(N>0\), we build the following function (for \(N=1,2, \ldots, 15)^{1}\).
\[
\Theta_{N}=C_{0} \cdot \phi_{1} \cdot \phi_{2} \ldots \phi_{N} \cdot D_{N}
\]
and check if it is satisfiable.

\subsection{Solution}
We define the above-mentioned variables at different N and find the CNF forms of each of the above. Next, we keep on adding our code for N = 1,2,3,4... and check the resultant output for each case until we get our desired output as P(N)=Q(N)=R(N)=S(N)=0.

\subsection{Code}
\begin{verbatim}
c   nvars nclauses
p cnf 10 22
c   -2 means (~x2), 2 means x2
c   X(0)=1, P(0)=2, Q(0)=3, R(0)=4, S(0)=5
c   X(1)=6, P(1)=7, Q(1)=8, R(1)=9, S(1)=10
c   At time N=0 (phi_0) 
c   P(0)
2 0 
c   Q(0)
3 0
c   R(0)
4 0
c   S(0) 
5 0
c   At time N=1 (phi_1) 
c   ~X(0) ~P(0) ~S(0) P(1)
-1 -2 -5 7 0
c   X(0) P(0) ~S(0) P1
1 2 -5 7 0
c   X(0) ~P(0) S(0) P(1)
1 -2 5 7 0
c   ~X(0) P(0) S(0) P(1)
-1 2 5 7 0
c   X(0) ~P(0) ~S(0) ~P(1)
1 -2 -5 -7 0
c   ~X(0) P(0) ~S(0) ~P1(1)
-1 2 -5 -7 0
c   ~X(0) ~P(0) S(0) ~P(1)
-1 -2 5 -7 0
c   X(0) P(0) S(0) ~P(1)
1 2 5 -7 0
c   Q(1) + ~P(0)
8 -2 0
c   ~Q(1) + P(0)
-8 2 0
c   R(1) + ~Q(0)
9 -3 0
c   ~R(1) + Q(0)
-9 3 0
c   S(1) + ~R(0)
10 -4 0
c   ~S(1) R(0)
-10 4 0
c   n=N=1 (D_n)
c   ~P(1)
-7 0
c   ~Q(1)
-8 0
c   ~R(1)
-9 0
c   ~S(1)
-10 0
\end{verbatim}
There is no values satisfying the above equation as we get its output as
\textbf{Output for N = 1}
    UNSAT\\
Similarly, we write code for N = 2,3,4 and observe the outputs for then:
\textbf{Output for N = 2}
    UNSAT\\
\textbf{Output for N = 3}
    UNSAT\\

For N=4, the following values are satisfying the conditions P(N)=Q(N)=R(N)=S(N)=0.
\begin{verbatim}
SAT
-1 2 3 4 5 6 -7 8 9 10 11 -12 -13 14 15 16 -17 -18 -19 20 -21 -22 -23 -24 -25 0
\end{verbatim}

\section{Conclusion}
Hereby the assignment is completed using the minisat solver.
\end{document}